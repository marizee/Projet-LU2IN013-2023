\documentclass[a4paper]{article}
\usepackage{amsmath, amssymb, mathtools}
\usepackage{amsthm}
\usepackage{fontspec}
\usepackage{xunicode}
\usepackage{fancyhdr}
\usepackage[french]{babel}
\usepackage[a4paper,centering]{geometry}
\usepackage{algorithm, algorithmic}
\usepackage{listings}
\usepackage{color}
\usepackage{graphicx}

\fancyhead[L, C, R]{}
\fancyfoot[L, R]{}
\fancyfoot[C]{\vspace{1mm}\thepage}
\renewcommand{\headrulewidth}{1pt}
\renewcommand{\footrulewidth}{1pt}
\pagestyle{fancy}

\definecolor{gray}{rgb}{0.94, 0.94, 0.94}
\definecolor{red}{rgb}{0.6, 0, 0}

\lstset{
    numbers=left,
    numberstyle=\tiny, 
    backgroundcolor=\color{gray},
    language=Python,
    keywordstyle=\color{red},
}

\begin{document}

\thispagestyle{plain}

\begin{titlepage}
    \begin{center}

        \bigskip
        \includegraphics[scale=0.5]{logo_su.jpg}~\\[4cm]

        {\LARGE Rapport du projet LU2IN013}\\[0.3cm]
        \rule{\linewidth}{0.5mm} \\[0.6cm]
        {\huge \textbf{Composition modulaire des polynômes à une variable}}\\[0.4cm]
        \rule{\linewidth}{0.5mm} \\[1cm]
        {\large Encadrant : M. Vincent NEIGER}\\[5cm]

        {\Large Serigne Fallou FALL, Marie BONBOIRE}
        
        \vfill
        Février 2023 - Juin 2023


    \end{center}
\end{titlepage}

\newpage

\tableofcontents

\newpage


\section{INTRODUCTION}
\subsection{Sujet}
\subsection{Objectifs}
\subsection{Choix d'implantation}


\section{NOTIONS}

\subsection{Airthmétique modulaire}

\subsection{Multiplication}

\begin{lstlisting}[frame=leftline, title={multiplication naive}]
    def mult_naive(f, g) :
    ring = f.parent()
  
    res = [0]*(f.degree()+g.degree()+1) 
    
    for i in range(0, f.degree()+1):
        for j in range(0, g.degree()+1):
            res[i+j] += f[i]*g[j]

    return ring(res) 
\end{lstlisting}

\begin{lstlisting}[frame=leftline, title={karastuba}]
def karatsuba(f, g) :
    ring = f.parent()
    t = ring.gen()
    if f.is_zero() : return f
    if g.is_zero() : return g
    
    if (f.degree() <= 10 and g.degree() <= 10) : 
        return mult_naive(f, g)

    k = max(f.degree()/2, g.degree()/2).ceil()

    f0 = ring(f.list()[:k]); f1 = ring(f.list()[k:])
    g0 = ring(g.list()[:k]); g1 = ring(g.list()[k:])

    h1 = karatsuba(f0, g0)
    h2 = karatsuba(f1, g1)
    h5 = karatsuba(f0+f1, g0+g1)
    h7 = h5 - h1 - h2

    h = h1 + h7.shift(k) + h2.shift(2*k)

    return h
\end{lstlisting}

Perfomances : inserer tableau de comparaison  \cite{aecf}

\section{ALGORITHMES DE COMPOSITION MODULAIRE}
\subsection{Algorithme naïf}

\begin{lstlisting}[frame=leftline, title={naive}]
def eval_naive_improved(g, a, f) :
	ring = g.parent()

	res = ring(g[0])
	ai = a % f
	for i in range(1, g.degree()+1) :
		res = res + g[i]*ai
		ai = (a*ai) % f
	return res % f
\end{lstlisting}

\subsection{Algorithme d'Horner}

\begin{lstlisting}[frame=leftline, title={Horner}]
def horner(g, a, f) :
    res = g[g.degree()]
    for i in range(g.degree()-1, -1, -1) :
        res = (res*a)%f + g[i]
    return res%f
\end{lstlisting}

\subsection{Algorithme de Brent et Kung}

\begin{lstlisting}[frame=leftline, title={brent and kung}]
def brentkung(g, a, f) :

	ring = a.parent()
	d = g.degree()+1; n = f.degree()
	if a.degree() >= n:
		raise ValueError("Erreur: a de degré trop élevé")

	r = RR(d.sqrt()).ceil()
	s = RR(d/r).ceil()

	ac = [ring(0)]*(r+1)
	ac[0] = ring(1)

	for i in range(1, r+1) :
		ac[i] = (a*ac[i-1]) % f

	ma = matrix(r, n, [(ac[i])[j] for i in range(r) for j in range(n)])
	mg = matrix(s, r, [g[i*r+j] for i in range(s) for j in range(r)])
	mb = mg*ma

	b = [ring(0)]*s
	for i in range(s) :
		b[i] = ring(mb[i].list())

	res = b[0]
	ar = ac[r]

	res = horner(b, ar, f)
	return res
\end{lstlisting}

\subsection{Comparaison}


\section{ALGORITHME DE NUSKEN ET ZIEGLER}



\newpage
\section{BIBLIOGRAPHIE}

\begin{thebibliography}{10}
    \bibitem{aecf} Alin Bostan, Frédéric Chyzak, Marc Giusti, Romain Lebreton, Grégoire Lecerf, Bruno Salvy, Eric Schost,
    \emph{Algorithmes efficaces en calcul formel}, 2017

\end{thebibliography}


\end{document}


