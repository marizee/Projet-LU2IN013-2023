\documentclass[a4paper]{article}
\usepackage{amsmath, amssymb, mathtools}
\usepackage{amsthm}
\usepackage{fontspec}
\usepackage{xunicode}
\usepackage[french]{babel}
\usepackage[left=2cm,right=2cm,top=2cm,bottom=2cm]{geometry}

\title{Composition modulaire}
\author{Fallou}

\begin{document}
\maketitle
\section{Préliminaires}
\newtheorem*{nt}{Notations}
\begin{nt}
Dans cette article, $\mathbb{K}$ est un corps arbitraire.
\\Les vecteurs, tels que les éléments de $\mathbb{A}^{m}$ ou $\mathbb{A}[\mathrm{X}]^{m}$ , sont par défaut, considérés comme des vecteurs colonnes; quand un tel vecteur colonne est considéré, il est explicité par la notation \text{e.g.} $\mathbb{A}^{1*m}$ ou $\mathbb{A}[\mathrm{X}]^{1*m}$.
\end{nt}
$\omega$ représente le plus petit nombre réel tels que deux matrices $\textit{n}\times\textit{n}$ peuvent être multipliées en $O(\textit{n}^{\omega+\epsilon})$ opérations pour tout $\epsilon>0$. La meilleure bonne existence aujourd'hui est $\omega<2.37286$. \text{Analogiquement} pour un produit de matrices $\textit{n}\times\textit{n}^{2}$ et $\textit{n}^{2}\times\textit{n}$, on a $\omega\approx3.25$ tel que le produit est en $O(\textit{n}^{\omega_{2}})$.
\section{Algorithmes de Brent et Kung}
\subsection{Composition modulaire}
Commençons par une revue de l'algorithme de \text{Brent et Kung} pour le calcul de \text{g(a)}  mod \text{f} en mettant en avant l'impact de la multiplication de matrices rectangulaires et à quel point la vitesse d'exécution dépend du degré de \text{f et de g}. Il peut  vu comme une introduction à l'algorithme de \text{Nüsken-Ziegler} qui généralise cette approche à un polynôme \text{g} à deux variables.
\theoremstyle{proposition}
\newtheorem*{pp}{Proposition}
\begin{pp} 
Étant donnés $\text{f}\in\mathbb{K}[\mathrm{X}]$ de degré n, a dans $\mathbb{K}[\mathrm{X}]_{<n}$ et \text{g} dans $\mathbb{K}[\mathrm{Y}]_{<n}$, l'algorithme de \text{Brent et Kung} calcule \text{g(a)}  mod \text{f} avec une complexité en \text{\~O}$((1+\text{n/d})d^{\omega_{2}/2})$.
\end{pp}
\textbf{Preuve:} La validité vient du fait qu'à l'étape 7 $b_{i}\equiv g_{ir} + g_{ir+1}a + ... + g_{ir+r-1}a^{r-1}$ mod \textit{f} pour tout i, avec $g_{j}$ le coefficient de degré j, dans g pour tout j. La complexité de l'algorithme vient du nombre $\theta(d^{1/2}$ de multiplications de modulo \textit{f}, qui utilisent \text{\~O}$(nd^{1/2})$ opérations dans $\mathbb{K}$ et d'un produit de matrices de taille $s\times r$ et $r\times n$, avec s et r en $\theta(d^{1/2})$. Ce produit peut être réalisé en $\lceil n/d \rceil$ $\leqslant n/d + 1$ produits de matrices $s\times r$ et $r\times d$, chacun de complexité $O(d^{\omega_{2}/2})$ opérations dans $\mathbb{K}$.
\theoremstyle{remark}
\newtheorem*{rem}{Remarque}
\begin{rem}
En analyse, diviser le produit matriciel en blocs, comme effectué, n'est pas la manière la plus optimale. En utilisant directement la multiplication de matrices rectangulaires, la vitesse d'exécution peut être estimée précisément en \text{\~O}$(d^{\omega_{2log_{d}n}/2})$. Il existe un algorithme qui multiplie une matrice de $n\times \lceil n^{\theta} \rceil$ par une matrice $\lceil n^{\theta} \rceil\times n$ en $O(n^{\omega_{\theta}})$ opérations, $\omega_{\theta}$ étant une puissance réalisable pour un produit de matrices rectangulaires pour n'importe quel nombre réel $\theta$. Cependant, ce raffinement complique les notations et ne sera pas utilisé pour nos résultats principaux. La même remarque vaut aussi pour d'autres estimations de vitesse d'exécution de cette section.
\end{rem}
\theoremstyle{algorithme}
\newtheorem*{al}{Algorithme}
\begin{al} 
Composition modulaire \text{Brent et Kung (f , a, g)}

\[
\fbox{
    \begin{minipage}[c]{11cm}
        \textbf{Entrée:} f de degré n dans $\mathbb{K}[\mathrm{X}]$, a dans $\mathbb{K}[\mathrm{X}]_{<n}$ et \text{g} dans $\mathbb{K}[\mathrm{Y}]_{<d}$\\
        \textbf{Sortie:} \text{g(a) rem f}
        \begin{enumerate}
        \item \text{r} $\leftarrow \lceil d^{1/2} \rceil$, \text{s} $\leftarrow \lceil d/r \rceil$ 
        \item $\text{â}_{0} \leftarrow 1$ \text{avec } $\text{â}_{i}=\text{a}^{i} \text{ rem f}$
        \item \text{for i=1,...,r do} $\text{â}_{i} \leftarrow \text{a}.\text{â}_{i} \text{-1 rem f}$
        \item \text{A} $\leftarrow \text{mat}(\text{coeff}(\text{â}_{i},j))_{0\leqslant \text{i}<\text{r}, 0\leqslant \text{j}<\text{n}}  dans \mathbb{K}^{r\times n}$
        \item \text{G} $\leftarrow \text{mat}(\text{coeff}(g,\text{ir+j}))_{0\leqslant \text{i}<\text{s}, 0\leqslant \text{j}<\text{r}}  dans \mathbb{K}^{s\times r}$
        \item \text{B = } $(\text{b}_{i,j})_{0\leqslant \text{i}<\text{r}, 0\leqslant \text{j}<\text{n}} \leftarrow \text{GA } \text{dans } \mathbb{K}^{s\times n}$
        \item \text{for i=0,...,s-1 do} $\text{b}_{i} \leftarrow \text{b}_{i,0}+...+\text{b}_{i,n-1}\text{x}^{n-1}$
        \item \text{return } $\text{b}_{0}+\text{b}_{1}\text{â}_{r}+...+\text{b}_{s-1}\text{a}^{s-1} \text{ rem f } \textbf{(Évaluation d'Horner)}$ 
        \end{enumerate}
    \end{minipage}
}
\]
\end{al}
\end{document}
